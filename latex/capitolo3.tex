\chapter{Technical Overview}
\label{chapter 3}
\thispagestyle{empty}


\noindent In questa sezione si deve descrivere l'obiettivo della ricerca, le problematiche affrontate ed eventuali definizioni preliminari nel caso la tesi sia di carattere teorico.

general introduction( i.e.: the spread of mobile phone, connectivity everywhere, data exchange.....)
The use of smartphone, tablet, laptop and other smart device has become fundamental in everyday life. Data traffic grows rapidly People need a connection, Wi-Fi or mobile, to work, study and so on. Also the data exchange bet
two main technologies: wifi and bluetooth
mac address
goal
how wifi and bt signals are related and in particular how wifi and bt mac address can be coupled each other

\section{Wi-Fi}
Wi-Fi is a technology for wireless local area networking with devices based on the IEEE 802.11 standards. In the US and in Europe, Wi-Fi operates at 2.4 GHz (802.11b/g) over 11 channels, three of which are not overlapping (1, 6, 11). Recently WiFi supports also 5 GHz (802.11n) with 21 channels with higher capacity, but a shorter range compared to 2.4 GHz. Modern device can switch between 2.4 GHz and 5 GHz, using a technique called \textit{band steering}, depending on traffic demand.\\
\linebreak
Every devices, with WiFi interface turned on, regularly broadcast Wi-Fi probe requests, in order to advertise their presence and actively discover Wi-Fi access points (AP) in proximity. This mechanism is called active scan and permits devices to have a list of near access point. IEEE 802.11 define another mechanism to discover WiFi AP: a passive mechanism, in which APs periodically advertise their presence to mobile devices using beacons.
\paragraph{Passive Scanning}
When a device performs passive scanning, it starts to listen over the 11 Wi-Fi channels hopping periodically from one to another passively detect nearby APs. When a beacon is captured, the mobile device responds with a Wi-Fi association frame. The beacons contain network configuration parameters, such as the Service Set Identifier (SSID), the type of encryption and the supported data rates. Most APs set a beaconing interval of about 100ms, but it depends on the hardware specification.\\
The main disadvantage of the passive scanning is listening on all 11 the channels, this operation is time consuming and do not ensure all the beacon are captured.
\paragraph{Active Scanning}
During the active scanning, the mobile device stimulates its nearby access points sending probe requests. The probe packet includes the device unique identifier, the device supported standards, the probe sequence number (SN) and other fields. The probe can be directed to all the AP (broadcast) or to a specific access point by indicating the SSID. \\
Active scanning is particularly helpful in scenarios where a mobile device roams across APs. It's also faster and less energy consuming than passive scanning.\\
Also, active scanning is the only method to connect to hidden network.
\subparagraph{probe request structure}
Figure number XY represent the packet structure of a probe request. The interesting fields are:
\begin{itemize}
\item Address 1: receiver MAC address (usually broadcast)
\item Address 2: sender MAC address
\item Address 3: AP MAC address (BSSID)
\item Sequence Control: the sequence number (SN) that represent a probe request
\item Payload: the list of the mobile devices SSID
\item FCS: a redundant check code 
\end{itemize}

In table number XY is shown a credible example of probe request. It follows the IEEE 802.11 standard so it's not encrypted. In our case, a device with MAC address xx:xx:xx:xx:xx is broadcasting a probe request with SSID "polimi-protected" and sequence number equal to 12.

\subparagraph{Probe request number}
The number of probe requests sent by a mobile phone is very variable among devices. On average some mobile devices send probe requests as often as 55 times per hour, but it might broadcast about 2000 probes per hour.(cit how talkative is your device).\\
The frequency of the probe request depends on:
\begin{itemize}
\item Wi-Fi chipset: the vendor can set up different parameters depending the company policies;
\item device operating system: the version of the OS and the device settings can affect the number of probe. For example, a fast speed connection setting can send an high number of probe or an energy saving mode can emit a low number of them;
\item frequency of screen unlocking: unlock the screen stimulates the probes activity.
\item number of applications: (forse)
\end{itemize}

\section{Bluetooth}
Bluetooth (IEEE 802.15.1) is a wireless technology standard for exchanging data over short distances from fixed and mobile devices, and building personal area networks (PANs).
Bluetooth was originated in 1994, when Jaap Haartsen, an electro technician employed at Ericsson, developed it in cooperation with Sven Mattisson. The name is based on the Danish word Bl\r{a}tand, the tenth-century king of Denmark and Norway. \\
The purpose of Bluetooth is replace cables with short-range and cheap radio connection that permits communication between mobile phones and peripherals.\\
Bluetooth is an open and royalty-free and, thanks to this, it is the standard for short-range wireless communication in WPAN (Wireless Personal Area Network) situations.
It operates in the universally unlicensed (but not unregulated) Industrial, Scientific and Medical (ISM) band at 2.4 GHz.
In the available frequency band, 79 sub-frequencies are used to transmit data, hopping from a frequency to another 1600 times per second in a pseudo random way.\\
\linebreak
The range of communication of Bluetooth and the transmission power are determined by their Class. As we can see in the table number XYZ Class 1 radios has the longest range of transmission (100 meters), instead Class 3 has a range of up to 1 meter. 
In this research, the used devices are mostly belonging to Class 2 (e.g. smart phones, tablets, laptops).
TABLE!!!
\\
Bluetooth architecture is based on master/slave model. A single master device can be connected up to seven different slave devices to generate a network, called piconet. The master shares his clock with the slaves and coordinates and manage the connection in the piconet. The master also sends and requests data to the slaves.

\subsection{find bt device (inquiry)}
In order to start Bluetooth connections between devices, the target device must be turned on and be visible. The device can be also turned on, but not be visible; in this case the pairing process is possible only if the target address is known.\\
To discovery visible devices, an inquiry mode has been defined. Basically, a device which wants to set up a Bluetooth connection with another one, sends out an inquiry packet and the other visible devices listening for them can answer.\\
A single Bluetooth inquiry scan process can last until 10.24 seconds(cit. BT and Wifi crowd data collection) and, at the end of the scan, zero or more devices can be discovered. \\
The inquiry scan, called \textit{Inquiry with RSSI}, contains information about:
\begin{itemize}
\item \textbf{device name:} the name that the owner assigns to the device;
\item \textbf{device profile:} type of the device (e.g.: phone, laptop, bluetooth headset, etc.);
\item \textbf{supported services:} the Bluetooth services provided by the device (e.g.: Advanced Audio Distribution Profile (A2DP), Audio Video Remote Control Profile (AVRCP), Basic Imaging Profile (BIP);
\item \textbf{unique MAC address:} a physical address assigned uniquely to each device;
\item \textbf{timestamp:} the date and the time of the discovery;
\item \textbf{signal strength (RSSI):} the measurement of the power present in a received radio signal (devo mettere la cit?).
\end{itemize}
trovare packet structure dell'inqury

\subsection{Bluez}
In the Linux kernel-based family operating system, the Bluetooth stack is managed by Bluez.
The most useful command of Bluez is \textit{hcitool}. Hcitool (Host Controller Interface Tool) is used to configure Bluetooth connections and send some special command to Bluetooth devices, e.g. inquire a remote device.
Also, hcitool provide access to the RSSI, the LQ and the TPL of a connected device, three fundamental status parameters.\\
To obtain the previously mentioned values an active connection between the master device and the slave is needed.

\paragraph{Received Signal Strength Indicator (RSSI):}
According to the Bluetooth Core Specification, the RSSI is an 8-bit signed integer that indicates the difference between the received (RX = real RSSI, non so come chiamarlo) power level and the Golden Receiver Power Range (GRPR). \\
Using the command \textit{hcitool rssi <bdaddr>} a value between +15dBm and -35dBm is obtained. \\
A positive RSSI value indicates how many dB the RSSI is above the upper limit; a negative value indicates how many dB the RSSI is below the lower limit. The value zero indicates that the RSSI is inside the Golden Receive Power Range.

\paragraph{Transmit Power Level (TPL):}
TPL is an 8-bit signed integer which specifies the Bluetooth module's transmit power level (in dBm). Every Bluetooth class has a fixed value and it doesn't change during a Bluetooth connection. For example, Class 2 devices has +4 dBm as maximum power, Class 3 has 0 dBm and Class 1 has +20 dBm.

\paragraph{Link Quality (LQ):}
Link Quality is a value from 0 to 255, which represents the quality of the link between two devices. The higher the value, the better the link quality is. For most Bluetooth modules, it is derived from the average bit error rate (BER) seen at the receiver, and is constantly updated as packets are received.

\subsubsection{inquiry vs rssi (golden range ...)}
As explained in section 3.XYZ, using hcitool of Bluez we can obtain two different types of RSSI values.
The first value is the RSSI obtained from the inquiry scan (\textit{inqury with RSSI}), the second one is the RSSI obtained directly from a connected device.\\
This two values are strictly related with a linear dependence; the relation is further analyzed in section X.Y.Z. 
To be clearer, from now on, the value obtained from the inquiry scan will be called RX. On the other hand, the value obtained from a connected device will be simply called RSSI.

\subsection{ping (l2ping) and echo time}
The Linux Bluetooth stack also permits to ping a Bluetooth device.\\
Ping is a utility used to test the reachability of an host, in our case a Bluetooth machine. It measures the round-trip time for messages sent from the originating host to a destination that are echoed back to the source.\\
In particular, for Bluetooth is used the command \textit{l2ping}. L2ping sends a L2CAP echo request to the Bluetooth MAC address (cit man page), and wait and echo response from the target device.
The ping feature is useful to understand if a Bluetooth device is in a particular range. If so, l2ping utility starts to send several echo requests to the target. If not, an error message is shown.\\

In particular, if the echo request is successful l2ping starts to ping the Bluetooth target device. In the default mode these fields are shown:\\
\begin{itemize}
\item the size of the single packet of the echo request (default 44 bytes)
\item the mac address of the target
\item the id of the packet
\item the echo response time (in milliseconds)
\end{itemize}

\section{Mac Address}
Mac Address is the acronym of Media Access Control Address. It is an unique identifier of a IEEE 802 network interface. Some examples of IEEE 802 standards are: ethernet, Wi-Fi, ZigBee, FDDI (Fiber Distributed Data Interface) and Bluetooth. \\
In our case MAC address is a fundamental information because it identifies uniquely a particular network interface of the device. Considering that a smartphone is equipped with Wi-Fi and Bluetooth chipset, a device is characterized by a couple of MAC Address: one MAC for the Wi-Fi interface and the other one for the Bluetooth interface.\\
In both cases the structure is the same: a 12 digit (48 bit or 6 bytes) address, usually written it the following three format: 
\begin{itemize}
\item MM:MM:MM:SS:SS:SS
\item MM-MM-MM-SS-SS-SS
\item MMM.MMM.SSS.SSS
\end{itemize}
The leftmost 6 digits (24 bits) called prefix is associated with the adapter manufacturer. Each vendor registers and obtains MAC prefixes as assigned by the IEEE. Vendors often possess many prefix numbers associated with their different products.\\
The rightmost digits of a MAC address represent an identification number for the specific device. Among all devices manufactured with the same vendor prefix, each is given their own unique 24-bit number.\\
A real example of MAC address of the same device is:
\begin{itemize}
\item Wi-Fi address: F4:E3:FB:85:53:1D
\item Bluetooth address: F4:E3:FB:A5:66:D8
\end{itemize}
In the example above the the vendors digits are the same, but often happens that the same device has two completely different Wi-Fi and Bluetooth prefixes.

\subsubsection{Privacy implications}
Due to the fact the MAC address identify uniquely a device, it can be used to identify a person. As explained in Section 2 (Related Work) this can rise a great privacy issue. In fact (as explained in section 3.x and 3.y toglibile) both Wi-Fi and Bluetooth address are easy to obtain: the first one is sent in clear with the probe request and the Bluetooth Address is visible during the inquiry scan.\\
But Wi-Fi MAC address and Bluetooth MAC address are completely unconnected. It is not possible to deduce if two MAC addresses are of the same device.\\
The goal of this thesis is find an algorithm to couple a Wi-Fi MAC address to a Bluetooth MAC Address and vice versa.
